% \thispagestyle{empty}

\documentclass[12pt, a4paper]{article}

\usepackage[top=3cm, bottom=2.5cm, left=2cm, right=2cm, headsep=1.25cm]{geometry}
\usepackage{lipsum}                     % lorem ipsum
\usepackage{graphicx}                   % изображения
\usepackage[english, russian]{babel}    % русская локализация
\usepackage{caption}
\usepackage{anyfontsize}
\usepackage{wrapfig}
\usepackage{tikz}
\usepackage{amsmath}
\usepackage{amssymb}
\usepackage{xcolor}
\usepackage{mathtools}
\usepackage{fancyhdr}

\setcounter{page}{59}

% МОИ КРАСИВЫЕ ЦВЕТА.
\definecolor{darkblue}{HTML}{265999}
\definecolor{oryellow}{HTML}{f3b319}
\definecolor{myred}{HTML}{d94c19}

% Создание НЕПОВТОРИМОГО колонтитула
\pagestyle{fancy}   % применение fancy стилей к странице
\fancyhf{}  % очистка всех хедеров и футеров
\fancyhead[L]{
    \begin{minipage}{0.64\textwidth}
        \hfill КНИГА I ПРЕДЛ. XXXIV. ТЕОРЕМА \hfill{\raisebox{-0.3em}{\thepage}}
    \end{minipage}
}
\fancy  % применение.
\renewcommand{\headrulewidth}{0pt}  % делаем линию колонтитула невидимой

% Макросы
\newcommand{\slet}[1]{{\tiny \sffamily #1}} % для стилизации названий геометрических точек 
\newcommand{\lwh}{2pt}

% определение стиля узлов по умолчанию
\tikzset{every node/.style={inner sep=2pt}}

\begin{document}

\begin{minipage}[t]{0.6\textwidth}
    \begin{minipage}{\textwidth}
        \begin{wrapfigure}{l}{0.18\linewidth}
            \vspace{-\baselineskip}
            \includegraphics[width=2cm]{PLetter.png}
        \end{wrapfigure}
        ротивоположные \textit{стороны и углы параллелограмма равны, а диагональ}
        \tikz[baseline=-0.3em]{
            \draw[black, line width=\lwh] (0,0) -- (1,0)
            \node[above] at (0,0) {\slet{A}}
            \node[above] at (1,0) {\slet{D}}
        }
        \textit{делит его на две равные части.}
    \end{minipage}
    
    \vspace{1cm}
    
    \begin{center}
        Поскольку 
        \begin{Bmatrix}
            \tikz[baseline=-1em]{
                \draw[darkblue, fill=darkblue] (0,0) -- (0:0.7) arc (0:-60:0.7) -- cycle;
                \node[above left] at (0,0) {\slet{A}};
                \node[right] at (0:0.7) {\slet{B}};
                \node[below] at (-60:0.7) {\slet{D}};
            } =
            \tikz[baseline=0.5em]{
                \draw[oryellow, fill=oryellow] (0,0) -- (0:-0.7) arc (180:120:0.7) -- cycle;
                \node[left] at (0:-0.7) {\slet{C}};
                \node[above left] at (120:0.7) {\slet{A}};
                \node[below right] at (0,0) {\slet{D}};
            } \\
            \tikz[baseline=-1em]{
                \draw[myred, fill=myred] (0,0) -- (-60:0.7) arc (-60:-105:0.7) -- cycle;
                \node[above] at (0,0) {\slet{A}};
                \node[below left] at (-105:0.7) {\slet{C}};
                \node[below right] at (-60:0.7) {\slet{D}};
            } = 
            \tikz[baseline=0.5em]{
                \draw[myred, fill=myred] (0,0) -- (75:0.7) arc (75:120:0.7) -- cycle;
                \node[below] at (0,0) {\slet{D}};
                \node[above left] at (120:0.7) {\slet{A}};
                \node[above right] at (75:0.7) {\slet{B}};
            }
        \end{Bmatrix}
        (пр. I.29) \\
        
        \vspace{5pt}
        и \tikz{
            \draw[black, line width=\lwh] (0,0) -- (1,0);
            \node[above] at (0,0) {\slet{A}};
            \node[above] at (1,0) {\slet{D}};
        } общая обоим треугольникам. \\
        
        \vspace{0.5cm}
        \therefore \begin{Bmatrix}
            \tikz{
                \draw[myred, line width=\lwh] (0,0) -- (1,0)
                \node[above] at (0,0) {\slet{A}}
                \node[above] at (1,0) {\slet{B}}
            } = 
            \tikz{
                \draw[myred, line width=\lwh, dashed] (0,0) -- (1,0)
                \node[above] at (0,0) {\slet{C}}
                \node[above] at (1,0) {\slet{D}}
            } \\
            \tikz{
                \draw[oryellow, line width=\lwh] (0,0) -- (1,0)
                \node[above] at (0,0) {\slet{A}}
                \node[above] at (1,0) {\slet{C}}
            } = 
            \tikz{
                \draw[darkblue, line width=\lwh, dashed] (0,0) -- (1,0)
                \node[above] at (0,0) {\slet{B}}
                \node[above] at (1,0) {\slet{D}}
            } \\
            \tikz[baseline=-1em]{
                \draw[black, fill=black] (0,0) -- (180:0.7) arc (180:255:0.7) -- cycle;
                \node[left] at (180:0.7) {\slet{A}}
                \node[below left] at (255:0.7) {\slet{D}}
                \node[above right] at (0,0) {\slet{B}}
            } = 
            \tikz[baseline=0.5em]{
                \draw[black, fill=black] (0,0) -- (0:0.7) arc (0:75:0.7) -- cycle;
                \node[right] at (0:0.7) {\slet{D}}
                \node[above right] at (75:0.7) {\slet{A}}
                \node[below left] at (0,0) {\slet{C}}
            }
        \end{Bmatrix}
        (пр. I.26) \\
        и \tikz[baseline=-1em]{
            \draw[myred, fill=myred] (0,0) -- (-60:0.7) arc (-60:-105:0.7) -- cycle;
            \draw[darkblue, fill=darkblue] (0,0) -- (0:0.7) arc (0:-60:0.7) -- cycle;
            \node[above left] at (0,0) {\slet{A}};
            \node[right] at (0:0.7) {\slet{B}};
            \node[below left] at (-105:0.7) {\slet{C}};
        } = 
        \tikz[baseline=0.5em]{
            \draw[oryellow, fill=oryellow] (0,0) -- (0:-0.7) arc (180:120:0.7) -- cycle;
            \draw[myred, fill=myred] (0,0) -- (75:0.7) arc (75:120:0.7) -- cycle;
            \node[below] at (0,0) {\slet{D}};
            \node[above right] at (75:0.7) {\slet{B}};
            \node[left] at (0:-0.7) {\slet{C}};
        }
        (акс. II)
    \end{center}
    \hspace{2em} Следовательно, противоположные стороны и углы параллелограмма равны. И, поскольку треугольники 
    \tikz[baseline=1.4em]{
        \draw[myred, line width=\lwh, dashed] (0,0) -- (1,0)
        \draw[oryellow, line width=\lwh] (0,0) -- (0.3,1.2)
        \draw[black, line width=\lwh] (0.3,1.2) -- (1,0)
        \node[below left] at (0,0) {\slet{C}}
        \node[below right] at (1,0) {\slet{D}}
        \node[above] at (0.3,1.2) {\slet{A}}
    } и 
    \tikz[baseline=-1.6em]{
        \draw[black, line width=\lwh] (0,0) -- (0.7,-1.2)
        \draw[myred, line width=\lwh] (0,0) -- (1,0)
        \draw[darkblue, line width=\lwh] (1,0) -- (0.7,-1.2)
        \node[above left] at (0,0) {\slet{A}}
        \node[above right] at (1,0) {\slet{B}}
        \node[below] at (0.7,-1.2) {\slet{D}}
    }
    равны во всех отношениях (пр. I.4), диагональ делит параллелограмм на две равные части.
    \begin{flushright}
        ч. т. д.
    \end{flushright}
\end{minipage}
\hfill
\begin{minipage}[t]{0.3\textwidth}
    \hfill \\
    
    \vspace{-1.5cm}
    \tikz{
        \coordinate (A) at (0.9, 3.6)
        \coordinate (B) at (3.9, 3.6)
        \coordinate (C) at (0, 0)
        \coordinate (D) at (3, 0)
    
        \draw[black, fill=black] (C) -- ++(0:0.7) arc (0:75:0.7) -- cycle;
        \draw[myred, fill=myred] (A) -- ++(-60:0.7) arc (-60:-105:0.7) -- cycle;
        \draw[darkblue, fill=darkblue] (A) -- ++(0:0.7) arc (0:-60:0.7) -- cycle;
        \draw[myred, fill=myred] (D) -- ++(75:0.7) arc (75:120:0.7) -- cycle;
        \draw[black, fill=black] (B) -- ++(180:0.7) arc (180:255:0.7) -- cycle;
        \draw[oryellow, fill=oryellow] (D) -- ++(0:-0.7) arc (180:120:0.7) -- cycle;

        \draw[black, line width=\lwh] (A) -- (D)
        \draw[myred, line width=\lwh, dashed] (C) -- (D)
        \draw[oryellow, line width=\lwh] (C) -- (A)
        \draw[darkblue, line width=\lwh] (D) -- (B)
        \draw[myred, line width=\lwh] (A) -- (B)

        \node[below left] at (C) {\slet{C}}
        \node[above left] at (A) {\slet{A}}
        \node[above right] at (B) {\slet{B}}
        \node[below right] at (D) {\slet{D}}
    }
        
    
\end{minipage}

\end{document}