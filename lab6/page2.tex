\begin{center}
\begin{table*}[t]
    \captionsetup{justification=raggedleft, singlelinecheck=false, position=above}
    \caption*{Т а б л и ц а}
    \centering
    \small
    \begin{tabular}{|>{\raggedleft}p{2cm} m{0.5cm}  m{0.8cm} m{0.8cm} m{0.8cm} m{0.8cm} m{0.8cm} m{0.8cm} m{0.8cm} c >{\centering\arraybackslash}m{1.3cm}|}
         \hline
          & & & & & & \multicolumn{4}{c}{Продолжительность} & \\
          & & \multicolumn{2}{c}{Восход} & \multicolumn{2}{c}{Заход} & \multicolumn{2}{c}{дня} & \multicolumn{2}{c}{ночи} & \\
           & & ч & мин & ч & мин & ч & мин & ч & мин & Разность мин \\
          \hline
          
          \multirow{7}{*}{Март\vspace{2cm}} & & & & & & & & & & \\
          & 17 & 6 & 41 & 18 & 37 & 11 & 56 & 12 & 04 & --08 \\
          & 18 & 6 & 38 & 18 & 39 & 12 & 01 & 11 & 59 & +02 \\
          & 19 & 6 & 36 & 18 & 41 & 12 & 05 & 11 & 55 & +10 \\
          & 20 & 6 & 34 & 18 & 43 & 12 & 09 & 11 & 51 & +18 \\
          & 21 & 6 & 31 & 18 & 45 & 12 & 14 & 11 & 46 & +28 \\
          & & & & & & & & & & \\

          \hline
          \multirow{7}{*}{Сентябрь\vspace{2cm}} & & & & & & & & & & \\ 
          & 23 & 6 & 16 & 18 & 28 & 12 & 12 & 11 & 48 & +24 \\
          & 24 & 6 & 18 & 18 & 25 & 12 & 07 & 11 & 53 & +14 \\
          & 25 & 6 & 20 & 18 & 22 & 12 & 02 & 11 & 58 & +04 \\
          & 26 & 6 & 22 & 18 & 20 & 11 & 58 & 12 & 02 & --04 \\
          & 27 & 6 & 24 & 18 & 17 & 11 & 53 & 12 & 07 & --14 \\
          & & & & & & & & & & \\
          
          \hline
    \end{tabular}
    \label{tab:table}
\end{table*}
\end{center}

\vspace{-20pt}
Лучи света от небесного тела, прежде чем попасть в глаз наблюдателя, проходят сквозь земную атмосферу, преломляясь в ней, как в призме. Так как плотность атмосферы увеличивается к поверхности Земли, лучи преломляются все сильнее и сильнее по мере приближения к Земле.

\lipsum[1-3]

\fancyhf{}
\fancyfoot[LE,RO]{\textbf{3*\hfill\thepage}}
\pagestyle{fancy}